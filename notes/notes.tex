\documentclass{article}
\usepackage{fullpage}
\usepackage{brian}

\DeclareMathOperator{\diagb}{diagb}
\DeclareMathOperator{\vectorize}{vec}
\renewcommand{\C}{\ensuremath{\mathbb{C}}}

\title{Shift-invariant sparse coding and dictionary learning}
\author{Brian McFee}

\begin{document}
\maketitle

\section{Introduction}

\subsection{Background: sparse coding}

Traditional sparse coding and dictionary learning is concerned with the following problem:
\begin{align}
\min_{D, \{a^i\}} \quad& \frac{1}{n} \sum_{i=1}^n \left(\frac{1}{2} \left\|x^i - Da^i\right\|_2^2 + \lambda \|a^i\|_1\right) \notag\\
\suchthat \forall k \in [m]:\quad& d_k\trans d_k \leq 1 \label{eq:sparsecoding}
\end{align}

Here, $x^i\in\R^{d}$ are a sample of $n$ vectors, $D\in\R^{d\times m}$ is the
\emph{codebook}, and $a^i\in\R^{m}$ is the vector of codeword activations for $x^i$.

Jointly optimizing the codebook $D$ and activations $a^i$ results in a factorization of the $X$ such that
each $a^i$ is sparse (\eg, depends on $k \ll m$ codewords).

After learning $D$, a new sample point $x \in \R^d$ can be encoded to $A(x) \in \R^m$ by solving a LASSO problem:
\begin{align}
A(x) \leftarrow \argmin_{a\in\R^m} \quad & \frac{1}{2} \left\| x - Da\right\|_2^2 + \lambda \|a\|_1.
\label{eq:sparse:encoding}
\end{align}

\autoref{eq:sparsecoding} is not jointly convex in $D$ and $a^i$, so a common approach is to alternate between solving for each $a^i$ with a fixed $D$ (via \autoref{eq:sparse:encoding}), 
and then optimize $D$ for a fixed set of activations.

\subsection{Shift-invariant sparse coding}

In shift-invariant (or convolutional) sparse coding, the goal is similar, but the encoding
function is now a set of convolutions, rather than dot products:
\begin{align}
A_*(x) \leftarrow \argmin_a \frac{1}{2} \left\|x - \sum_{k=1}^m d_k * a_k \right\|_2^2 + \lambda
\sum_{k=1}^m \|a_k\|_1 \label{eq:sisc}
\end{align}
and the joint codebook learning problem is formulated similarly:
\begin{align}
\min_{D, \{a^i\}_{i=1}^n} \quad & \frac{1}{n}\sum_{i=1}^n \left( \frac{1}{2} \left\| x^i - \sum_{k=1}^m
d_k * a^i_k \right\|_2^2 + \lambda \sum_{k=1}^m \left\|a^i_k\right\|_1 \right) \notag\\
\suchthat \forall k \in [m]:\quad& d_k\trans d_k \leq 1
\end{align}

Note that $a^i \in \R^{k \times d}$ is a matrix of activations for each data point $x^i$: $a^i_{kt}$ indicates the (convolutional) weight of basis filter $d_k$ at time $t$ when reconstructing $x^i$.

\subsection{Shift-invariant group-sparse coding}
In audio applications, and especially music, element-wise sparsity may not be appropriate: it is reasonable to expect a song $x$ to repeatedly use a small subset of basis elements.  This
suggests replacing the $\ell_1$ penalty with a group-sparsity penalty, such as the mixed-norm $\ell_{1,2}$.  This would transform \autoref{eq:sisc} into the following problem:
\begin{align}
A_g(x) \leftarrow \argmin_a \frac{1}{2} \left\|x - \sum_{k=1}^m d_k * a_k \right\|_2^2 + \lambda
\sum_{k=1}^m \|a_k\|_2 \label{eq:sigsc}
\end{align}

More generally, we can consider arbitrary (convex) regularizers $g(\cdot)$:
\begin{align}
A(x) \leftarrow \argmin_a \frac{1}{2} \left\|x - \sum_{k=1}^m d_k * a_k \right\|_2^2 + g(a) \label{eq:arb}
\end{align}

\subsection{The frequency domain}
Convolutions are costly in the time domain (as in \autoref{eq:sisc}--\autoref{eq:arb}).  
We may instead formulate the problem in the frequency domain after applying the 
discrete Fourier transform $\FT:\R^d\rightarrow\C^d$.\footnote{To simplify notation, we will assume the symmetrically scaled version of the Fourier transform: $\forall_x: \|x\|_2 =
\|\FT\{x\}\|_2$.}\footnote{In general, I will assume that all signals have been appropriately padded to accommodate
convolution at the boundaries.}

By Parseval's theorem, linearity of $\FT$, and the convolution theorem, the $\ell_2$ error term of the above problems is preserved:
\begin{align*}
\left\|x - \sum_k d_k * a_k\right\|_2 &= \left\|\FT\left\{x - \sum_k d_k * a_k\right\}\right\|_2\\
                                    &= \left\|\FT\left\{x\right\} - \sum_k \FT\left\{d_k * a_k\right\}\right\|_2\\
                                    &= \left\|\FT\left\{x\right\} - \sum_k \FT\left\{d_k\right\} \circ \FT\left\{a_k\right\}\right\|_2,
\end{align*}
where $\circ$ denotes the Hadamard (element-wise) product.

With a slight abuse of notation to ease readability, we can reformulate \autoref{eq:sigsc}
entirely in the frequency domain:

\begin{align}
A_g(X) \leftarrow \argmin_A \frac{1}{2}\left\|X - \sum_{k=1}^m D_k \circ A_k\right\|_2^2 + \lambda \sum_{k=1}^m \|A_k\|_2 \label{eq:csigsc}
\end{align}
where $X, D_k, A_k \in \C^d$ now denote the frequency domain representations of the input signal, basis elements, and
activations.

The following lemma will be handy throughout the remainder of this note.
\begin{lemma}
There exist matrices $U, V$ such that $UV = \sum_{k=1}^m D_k \circ A_k$.
\label{lemma:rearrange}
\end{lemma}
\begin{proof}
$D$ and $A$ can be rearranged as follows:
\begin{align*}
U \leftarrow \diagb(D)     &\defeq \left[ \diag(D_1), \diag(D_2), \cdots, \diag(D_m)\right] \in \C^{d \times dm}\\
V \leftarrow \vectorize(A) &\defeq \left[ A_1\trans, A_2\trans, \cdots, A_m\trans\right]\trans \in \C^{dm}.
\end{align*}
\end{proof}

This allows us to reformulate \autoref{eq:csigsc} in a more familiar factorization setting:
\[
A_g(X) \leftarrow \argmin_A \frac{1}{2} \left\|X - \diagb(D)\vectorize(A)\right\|_2^2 + g(A).
\]

Direct optimization on complex variables can be tricky when the function is not complex-differentiable (as is the case for
$\ell_2^2$ error).  Instead, we will reparameterize in terms of the real and imaginary components:
\[
\C^d \ni X \mapsto \left[\begin{array}{r}\Re X \\ \Im X \end{array}\right] \in \R^{2d}
\]

This leads to the following optimization problem over real-valued $C$:
\begin{align}
A(X) \leftarrow \argmin_{C\in\R^{2dm}} \frac{1}{2} \left\| X - QC\right\|_2^2 + g(C),\label{eq:sigsc:realized}
\end{align}
where
\begin{align*}
C &\defeq \left[\begin{array}{l}\vectorize(\Re A)\\\vectorize(\Im A)\end{array}\right] \in \R^{2dm}\\
Q &\defeq \left[\begin{array}{rr}\diagb(\Re D) & -\diagb(\Im D)\\ \diagb(\Im D)& \diagb(\Re D)\end{array}\right] \in \R^{2d\times 2dm}.
\end{align*}

From an optimal $\overline{C}$, the complex-valued activations can be recovered by reshaping the matrix
\[
\overline{A} = \left[I_{dm}, \quad j I_{dm}\right] \overline{C} \in \C^{dm}
\]
into a $d\times m$ complex-valued activation matrix.

\section{A-step (encoder) via ADMM}
To solve \autoref{eq:sigsc:realized}, we'll apply the ADMM technique by introducing an auxiliary variable $Z \in \R^{2dm}$ and equality constraint:
\begin{align}
\min_{C, Z} \quad & \frac{1}{2} \left\|X - QC\right\|_2^2 + g(Z)\notag\\
\suchthat \quad & C - Z = 0.\label{eq:sigsc:admm}
\end{align}
Now, we will introduce a scaled dual variable $\Omega \in \R^{2dm}$ and scaling parameter $\rho>0$ to form the augmented Lagrangian:
\begin{align}
\Ell_\rho(C, Z, \Omega) = \frac{1}{2}\left\|X - QC\right\|_2^2 + \frac{\rho}{2}\left\|C - (Z - \Omega)\right\|_2^2 + g(Z).\label{eq:sigsc:auglag}
\end{align}
We then alternate over optimizing $C, Z, \Omega$ sequentially:
\begin{enumerate}
\item $C^{t+1} \leftarrow \argmin_C \Ell_\rho(C, Z^t, \Omega^t)$,
\item $Z^{t+1} \leftarrow \argmin_Z \Ell_\rho(C^{t+1}, Z, \Omega^t)$,
\item $\Omega^{t+1} \leftarrow \Omega^{t} + C^{t+1} - Z^{t+1}$.
\end{enumerate}

\subsection{C-step: ridge regression}
Note that the first step no longer depends on the regularizer $g$.  The remaining terms are
quadratic in $C$, and amount to a ridge-regression.  To see this, we can calculate the gradient and set to zero:
as
\begin{align*}
\mathbf{0} = \nabla_C \Ell_\rho(C, Z, \Omega) &= -Q\trans (X - QC) + \rho\left(C - (Z - \Omega) \right)\\
\Rightarrow \quad \overline{C} &= (Q\trans Q + \rho I)^{-1} \left(Q\trans X + \rho (Z - \Omega)\right).
\end{align*}

% This optimization superficially involves an extremely high-dimensional matrix inversion (or least-squares solution): $Q\trans Q + \rho I \in \R^{2dm \times 2dm}$.
% However, the matrix is highly structured, and can be partitioned into $d\times d$ diagonal sub-matrices of the form $d_i \circ d_j$, $d_i \circ d_i + \rho I$, or $-e_i \circ e_j$.
% It can therefore be permuted into a block-diagonal matrix and inverted block-wise:
% \begin{align*}
% \left(Q\trans Q + \rho I\right)^{-1} &= \left(\Pi\trans \Pi\left( Q\trans Q + \rho I\right)\Pi\trans \Pi \right)^{-1}\\
%                                      &= \Pi\trans \left( \Pi \left( Q\trans Q + \rho I\right)\Pi\trans \right)^{-1} \Pi,
% \end{align*}
% where $\Pi$ is the following permutation matrix:
% \[
% \Pi_{i,j} = \begin{cases}
% FIXME:  2013-02-20 20:53:44 by Brian McFee <brm2132@columbia.edu>
% this might be 0-indexed as i+ 
% Python code:
% P = zeros( (2 * d * m, 2 * d * m) )
% for i in range(d):
%   for j in range(2 * m):
%       P[2 * m * i + j, d * j + i] = 1
% 1 & \exists u\in [d], v \in [2m]: \quad (i = 2m(u-1) + v) \wedge (j = d(v-1) + i)\\
% 0 & \text{otherwise}
% \end{cases}
% \]
% (In fact, it will be simpler to optimize under the alternative parameterization $(Q, C) \mapsto (Q\Pi\trans, \Pi C)$ so that no permutations are necessary, although it complicates the notation
% in the derivation above.)

% Each $2m\times 2m$ diagonal block of $(\Pi Q\trans Q\Pi\trans + \rho I)$ can be inverted efficiently and in parallel, and then recombined by inverting (transposing) the permutation.  
% The block-wise inversion costs $\bigO(dm^3)$ (for $d$ inversions of $2m\times 2m$-matrices), rather than the $\bigO(d^3m^3)$ cost of the full matrix inversion.  
% Since convolutional dictionary learning encodes shift invariance, we should be able to get away with a much smaller dictionary than in the standard sparse coding setting, so that $m\ll d$, and
% the $C$-step can be solved in approximately (practically) linear time.

% Note that for a fixed dictionary and scaling $\rho$, this inverse can be pre-computed and cached offline as it does not depend on the current data point $X$.
% However, in practice, we may wish to adaptively scale $\rho$ in each iteration of the ADMM algorithm.

Note that the matrix being inverted is the sum of a diagonal ($\rho I$) and product of low-rank matrices
($Q\trans Q$).  Applying the matrix inversion lemma, we get:
\[
\left( Q\trans Q + \rho I \right)^{-1} = \rho^{-1} I - \rho^{-2} Q\trans \left(I + \rho^{-1} QQ\trans \right)^{-1} Q.
\]
The structure of $Q$ allows us to further simplify the inversion as follows.

\begin{lemma}
If $A = \diagb(\alpha)$ is the diagonal-block matrix associated with a matrix $\alpha$, and $B = \diagb(\beta)$ of compatible dimension, then $AB\trans = I\circ \left(\alpha\beta\trans\right)$.\label{lemma:diagonal}
\end{lemma}
\begin{proof}
By construction of $A$, each pair of rows $A_i, A_j$ is orthogonal.  Because the diagonal pattern is consistent for $A$ and $B$, we also have that $A_i B\trans_j = 0$ whenever $i \neq j$.
Along the diagonal, we have
\[
[AB\trans]_{ii} = \sum_k A_{ik} B_{ik} = \sum_k \alpha_{ik} \beta_{ik} = \alpha_i\trans \beta_i = [\alpha\beta\trans]_{ii}.
\]
\end{proof}
\begin{lemma}
If $A=\diagb(\alpha)$ and $B=\diagb(\beta)$, then ${AB\trans = BA\trans}$.\label{cor:symmetry}
\end{lemma}
\begin{proof}
Since both $AB\trans$ and $BA\trans$ are diagonal, we need only consider the diagonal entries. 
Following the argument in the proof of \autoref{lemma:diagonal}, we have the following equalities:
\begin{align*}
[AB\trans]_{ii} &= \alpha_i\trans \beta_i\\
[BA\trans]_{ii} &= \beta_i\trans \alpha_i,
\end{align*}
which are equivalent for $\alpha, \beta \in \R^d$, so $[AB\trans]_{ii} = [BA\trans]_{ii}$.
\end{proof}

\begin{theorem}
Let $Q\in\R^{2d\times 2dm}$ be the structured diagonal-block matrix as defined in \autoref{eq:sigsc:realized}.  Then $QQ\trans \in \R^{2d\times 2d}$ is diagonal.
\label{thm:diagb}
\end{theorem}
\begin{proof}
\begin{align*}
Q &= \left[\begin{array}{rr} A & -B\\ B & A \end{array} \right] & \text{by definition}\\
\Rightarrow \quad QQ\trans &= \left[\begin{array}{rr} A & -B\\ B & A\end{array} \right]\left[\begin{array}{rr} A\trans & B\trans\\ -B\trans & A\trans\end{array} \right]\\
&= \left[\begin{array}{rr} AA\trans  + BB\trans, & AB\trans - BA\trans\\ BA\trans - AB\trans, & AA\trans + BB\trans \end{array}\right]\\
&= \left[\begin{array}{rr} AA\trans  + BB\trans & 0 \\ 0 & AA\trans + BB\trans \end{array}\right] & \autoref{cor:symmetry}
\end{align*}
for some diagonal-block matrices $A$ and $B$. \xx{\autoref{cor:symmetry} could be deduced directly from symmetry of $QQ\trans$.}
By \autoref{lemma:diagonal}, $AA\trans$ and $BB\trans$ are diagonal, and so is their sum.  Therefore, $QQ\trans$ is also diagonal.
\end{proof}
It follows from \autoref{thm:diagb} that $I + \rho^{-1}QQ\trans$ is also diagonal, and its inverse $R\defeq\left(I + \rho^{-1}QQ\trans\right)^{-1}$ can be computed in linear time.

Intuitively, the diagonal of $AA\trans + BB\trans$ is $d$-dimensional vector, where the $i$th coordinate measures the total magnitude (squared) of the $i$th dimension across all codewords.

The final equation to solve for $\overline{C}$ is
\begin{align}
\overline{C} = \rho^{-1}\left( I - \rho^{-1} Q\trans R Q\right) (Q\trans X + \rho(Z-\Omega)).\label{eq:cstep}
\end{align}


% \begin{corollary}
% Let $Q$ as in \autoref{thm:diagb}.  Then there exists a permutation $\Pi$ such that $\Pi QQ\trans \Pi\trans$ is block-diagonal, and each block is $2\times2$.
% \label{cor:permute}
% \end{corollary}
% \begin{proof}
% Each row (column) of $QQ\trans$ has at most two non-zero entries.  Specifically, row (column) $i$ may have non-zeros only at column (row) $i$ (due to the diagonal block) and $i\pm d$ (due to the
% off-diagonal block).  The permutation $\Pi$ derived from the mapping
% \[
% \pi(i) \defeq \begin{cases}
% 2i - 1 & i \leq d\\
% 2(i - d) & i > d
% \end{cases}
% \]
% produces block-diagonal structure.
% \end{proof}
% Comment: $\Pi X$ for $X = [\Re X' ; \Im X']$ interleaves the real and imaginary components, rather than stacks them separately.

% For any permutation $\Pi$, we have the following identity:
% \begin{align*}
% \left(\Pi\trans\Pi (I + \rho^{-1} QQ\trans)\Pi\trans\Pi\right)^{-1}
% &= \left(\Pi\trans(I + \rho^{-1} \Pi QQ\trans\Pi\trans)\Pi\right)^{-1}\\
% &= \Pi\trans \left(I + \rho^{-1} \Pi QQ\trans\Pi\trans\right)^{-1}\Pi.
% \end{align*}
% In particular, \autoref{cor:permute} shows that $QQ\trans$ can be rearranged into $d$ blocks $q^i$, where each block takes the form
% \[
% q^i = \left[\begin{array}{rr}
% (AA\trans)_{ii} & - (BB\trans)_{ii}\\
% - (BB\trans)_{ii}& (AA\trans)_{ii}
% \end{array}\right].
% \]
% Consequently, $I + \rho^{-1}\Pi QQ\trans \Pi\trans$ is positive-definite and block diagonal, and therefore has a sparse, block-diagonal, and lower-triangular Cholesky factor $L$.

% Reassembling all the terms and defining $b \defeq Q\trans X + \rho(Z-\Omega)$, we have the following system to solve:
% \begin{align*}
% \overline{C} &= (Q\trans Q + \rho I)^{-1} b\\
% &= \left(\rho^{-1}I - \rho^{-2} Q\trans(I + \rho^{-1}QQ\trans)^{-1} Q \right) b\\
% &= \left(\rho^{-1}I - \rho^{-2} Q\trans\Pi\trans(I + \rho^{-1}\Pi QQ\trans\Pi\trans)^{-1} \Pi Q \right) b\\
% &= \rho^{-1} b - \rho^{-2} Q\trans\Pi\trans(I + \rho^{-1}\Pi QQ\trans\Pi\trans)^{-1}\Pi Q b.
% \end{align*}
% This leads to the following forward/back-substitution algorithm for $\overline{C}$:
% \footnote{Note that $Q$ always appears coupled with $\Pi$; in practice, it will be simpler to use the alternate encoding $X \mapsto \Pi X$ which interleaves real and imaginary components, which will
% eliminate all permutations from the equations and work directly with block-diagonal matrices.}
% \begin{enumerate}
% \item $L \leftarrow \text{chol}(I + \rho^{-1}\Pi QQ\trans\Pi\trans)$
% \item $\gamma \leftarrow L^{-1} \Pi Qb$
% \item $\overline{C} \leftarrow \rho^{-1}b - \rho^{-2} Q\trans\Pi\trans L^{-\mathsf{T}}\gamma$
% \end{enumerate}
% The payoff here is that the Cholesky factor $L$ can be computed in linear ($\bigO(d)$) time, since each diagonal block has constant size.  
% Similarly, all inversions are applied to $L$ (or $L\trans$), and again become linear-time operations. 

\subsection{Z-step: proximal mapping}
For a fixed $C^{t+1}$ and $\Omega^t$, step 2 above depends only upon the quadratic term and $g$.  In
particular, it takes the from of a $\prox$ update:
\begin{align}
Z^{t+1} &\leftarrow \argmin_Z g(Z) + \frac{\rho}{2} \|Z - (C^{t+1} + \Omega^t)\|_2^2 \label{eq:zstep}\\
        &= \prox_{\rho^{-1}g}(C^{t+1} + \Omega^t)
\end{align}

For the group-sparsity regularizer described above, the update takes the form
\begin{align*}
g(C) &= \lambda \sum_{k=1}^m \|C_k\|_2\\
\Rightarrow \quad Z_k^{t+1} &\leftarrow (C_k^{t+1} + \Omega_k^t) 
\left(1 - \frac{\lambda}{\rho \|C_k^{t+1} + \Omega^t_k\|_2}\right)_+
\end{align*}
where the $k$-subscript indicates the sub-vector corresponding to the $k$th codeword.

\subsection{Z-step continued: time and frequency}

Sometimes the regularizer $g$ is more naturally expressed as a time domain function (\eg, point-wise sparsity, or temporal continuity).  
More precisely, the following lemma follows by composition of convex and affine functions.
\begin{lemma}
Let $h : \C^n \rightarrow \R$ be a convex function, and $g(Z) \defeq h(\FT^{-1}\left\{Z\right\})$.  Then $g$ is convex in $Z$.
\label{ftconvex}
\end{lemma}
\begin{proof}
Let $\sum_i \lambda_i Z_i$ be a convex combination of points $Z_i$.  Then,
\begin{align*}
g\left(\sum_i \lambda_i Z_i \right) &= h\left(\FT^{-1}\left\{\sum_i \lambda_i Z_i \right\} \right) & \text{by definition of } g\\
&= h\left( \sum_i \lambda_i \FT^{-1}\left\{ Z_i\right\}\right) & \text{by linearity of } \FT^{-1}\\
&\leq \sum_i \lambda_i h\left(\FT^{-1}\left\{Z_i\right\} \right) & \text{by convexity of } h\\
&= \sum_i \lambda_i g(Z_i) & \text{by definition of } g.
\end{align*}
\end{proof}
By \autoref{ftconvex}, $g$ is admissible in the ADMM framework.
In a direct optimization framework (\eg, gradient descent on \autoref{eq:sigsc:realized}), operating
simultaneously on $Z$ and its (inverse) Fourier transform would be a significant nuisance.  However, the ADMM approach
greatly simplifies the process.

Note that \autoref{eq:zstep} depends on $C^{t+1} + \Omega^{t}$ only through the $\ell_2$-norm.
By Parseval's theorem and \autoref{eq:gtoh}, we have for any $U$:
\begin{align*}
Z^* 
&\defeq \argmin_Z g(Z) + \frac{\rho}{2} \left\|Z - U \right\|_2^2 \\
&= \argmin_Z h\left(\FT^{-1}\left\{Z\right\}\right) + \frac{\rho}{2} \left\|Z - U \right\|_2^2\\
&= \argmin_Z h\left(\FT^{-1}\left\{Z\right\}\right) + \frac{\rho}{2} \left\|\FT^{-1}\left\{Z - U\right\} \right\|_2^2\\
&= \FT\left\{\argmin_z h\left(z\right) + \frac{\rho}{2} \left\|z - \FT^{-1}\left\{U\right\}
\right\|_2^2\right\}\\
&= \FT\left\{\prox_{\rho^{-1} h} \left( \FT^{-1}\left\{U\right\}\right)\right\},
\end{align*}
where $z \defeq \FT^{-1}\left\{Z\right\} \Leftrightarrow \FT\{z\} \defeq Z$.

This suggests the following recipe for the $Z$ update:
\begin{enumerate}
\item $u \leftarrow \FT^{-1}\left\{C^{t+1} + \Omega^t \right\}$
\item $z \leftarrow \prox_{\rho^{-1}h} \left(u\right)$
\item $Z^{t+1} \leftarrow \FT\left\{z\right\}$
\end{enumerate}
The advantage of this approach is that the Fourier transforms need only be computed infrequently, and it's decoupled from
the $C$-step which is solved entirely in the frequency domain.

\section{D-step (dictionary learning)}
Given a sample of $n$ data points $\{X_i \given i \in [n] \} \subset \R^{2d}$ and their activations 
$\{C^i \given i \in [n] \} \subset \R^{2dm}$, the dictionary learning problem can be posed as the following
optimization:
\begin{align}
\min_{D\in \C^{d\times m}} \quad&\frac{1}{n} \sum_{i=1}^n \left(\frac{1}{2} \left\|X_i - \sum_{k=1}^m A^i_k
\circ D_k \right\|_2^2\right)\label{eq:siscd}\\
\suchthat \forall k: \quad & D_k^H D_k \leq 1.\notag
\end{align}

By applying \autoref{lemma:rearrange}, we can equivalently solve the corresponding (real-valued) linear regression problem:
\begin{align}
\min_{F\in \R^{2dm}} \quad& \sum_{i=1}^n \left(\frac{1}{2} \left\|X_i - S_i F\right\|_2^2\right) + g(F)\label{eq:siscdmf}
\end{align}
where
\begin{align*}
F &\defeq \left[\begin{array}{c}
\vectorize(\Re D)\\
\vectorize(\Im D)
\end{array}\right] \in \R^{2dm}\\
S_i &\defeq \left[\begin{array}{rr}
\diagb(\Re A^i) & -\diagb(\Im A^i)\\
\diagb(\Im A^i) & \diagb(\Re A^i)
\end{array}\right] \in \R^{2d \times 2dm}
\end{align*}
and $g(\cdot)$ encodes the feasibility constraint on $F$.

Again, we separate out the feasibility constraint by creating an auxiliary variable $E$ and Lagrangian parameter $W$, which results in the augmented Lagrangian:
\[
\Ell_\rho(F,E,W) \defeq \sum_{i=1}^n \frac{1}{2} \left\| X_i - S_i F\right\|_2^2 + g(E) + \frac{\rho}{2} \left\| F - (E - W)\right\|_2^2.
\]
\subsection{F-step via ADMM}
To compute $\overline{F}$ given $E$ and $W$, we again compute the gradient:
\begin{align*}
\nabla_F \Ell_\rho(F, E, W) &= \sum_{i=1}^n -S_i\trans (X_i - S_i F) + \rho (F - (E - W)
)\\
&= 0\\
\Rightarrow \quad \overline{F} &= \left(\rho I + \sum_{i=1}^n S_i\trans S_i
\right)^{-1} \left(\sum_{i=1}^n S_i\trans X_i + \rho(E - W) \right).
\end{align*}
Like the $A$ update, the $F$ update takes the form of a ridge regression.
Unfortunately, since the $2dm\times 2dm$ matrix being inverted is not necessarily low-rank, we cannot
directly apply the matrix inversion lemma here.
% \subsection{D-step via ADMM}
% Although \autoref{eq:siscdmf} resembles the regression problem encountered during the $C$-step, the involvement of multiple encodings $S_i$ will render the matrix inversion lemma approach
% untenable.  Instead, we will split the dictionary along examples, as follows:
% \begin{align}
% \min_{F, F_i} \quad& \sum_{i=1}^n \frac{1}{2} \left\|X_i - S_i F_i\right\|_2^2 + g(F)\\
% \suchthat \forall i: \quad & F_i - F = 0 \notag.
% \end{align}
% This results in the augmented Lagrangian:
% \[
% \Ell_\rho(F_i, F, E_i) = \sum_{i=1}^n \frac{1}{2} \left\|X_i - S_i F_i\right\|_2^2 + g(F) + \frac{\rho}{2}\sum_{i=1}^n \|F_i - (F - E_i)\|_2^2,
% \]
% where we now have a dual parameter $E_i$ for each constraint $F_i$.

% \subsection{$F_i$-step}
% To solve for each data point's optimal codebook, we compute the gradient:
% \begin{align*}
% \zero = \nabla_{F_i}\Ell_\rho &= -(S_i)\trans (X_i - S_i F_i) + \rho(F_i - (F - E_i)\\
% \Rightarrow\quad \overline{F_i} &= \left(\rho I + (S_i)\trans S_i \right)^{-1} \left((S_i)\trans X_i + \rho (F - E_i) \right).
% \end{align*}
% This equation is identical to the derivation of the optimal $C$-step.  Moreover, the $S_i$ matrix exhibits the same structure as $Q$, and we can thus apply the matrix inversion lemma, 
% \autoref{thm:diagb}, and \autoref{eq:cstep} to solve for $F_i$ efficiently by inverting only diagonal matrices.  The update then takes the form:
% \begin{align}
% \overline{F_i} = \rho^{-1}\left( I - \rho^{-1} (S_i)\trans \Sigma S_i\right) ((S_i)\trans X_i + \rho(F-E_i)).\label{eq:fistep}
% \end{align}
% with $\Sigma \defeq \left(I + \rho^{-1} (S_i) (S_i)\trans\right)^{-1}$ diagonal.

% Note that because the $F_i$ are all decoupled for a fixed $F$, the codebooks may be solved in parallel, and each full ADMM cycle can be parallelized as a simple scatter/gather (map/reduce)
% pattern.

% \subsection{$F$-step}
% To solve for $F$, we optimize $\Ell_\rho$ again:
% \[
% \overline{F} = \argmin_F g(F) + \frac{\rho}{2} \sum_{i=1}^n \|F - (F_i + E_i)\|_2^2.
% \]
% Since $g(\cdot)$ is the indicator of a convex set $\sigma$, this update is equivalent to the projection:
% \begin{align}
% \overline{F} = \Pi_\sigma\left[\sum_{i=1}^n F_i + E_i \right], \label{eq:fstep}
% \end{align}
% which for the case of $\ell_2^2$ bounds on each basis element, reduces to simply scaling each basis down to at most unit length.\xx{Apparently this follows from Pshenichnyi-Rockafellar theorem;
% see also Borwein \& Lewis exercise 3.13c.}

% To verify that \autoref{eq:fstep} is correct, first observe that the objective and constraints are separable over the subspaces corresponding to individual codewords.  
% It then suffices to solve the following simplified problem (implicitly on each subspace):
% \[
% \min_f \sum_i \frac{1}{2} \|f - m_i\|^2 \quad \suchthat \quad f\trans f \leq 1.
% \]
% This can be broken into two cases.  Let $\mu \defeq \nicefrac{1}{n}\sum_i m_i$.  If $\mu\trans \mu \leq 1$, then $f = \mu$ is optimal because it feasible, and also the solution to the unconstrained problem.
% Otherwise, it can be shown analytically that the optimal $f$ takes the form $\overline{f} = \mu/\|\mu\|$, which is exactly the projection of $\mu$ onto the unit ball.

\section{Implementation}

Assembling all of the pieces, the dictionary learning algorithm is given as \autoref{algorithm}.

\begin{algorithm}
\caption{CDL$(X, g, m)$\hfill Convolutional dictionary learning\label{algorithm}}
\begin{algorithmic}[]
\REQUIRE{Input data (frequency representation) $X \in \C^{d\times n}$, regularization function $g(\cdot)$, dictionary size $m>0$}
\ENSURE{Convolutional dictionary $D \in \C^{d\times m}$}
\item[]{$X \leftarrow \left[\begin{array}{c}\vectorize(\Re X)\\ \vectorize(\Im X) \end{array}\right]$}
\item[]{Initialize $D \leftarrow \diagb(\text{random}(2d, m))$}
\FOR{$T= 0, 1, 2, \cdots$}
    \item $DX \leftarrow D\trans X,\quad \Delta = \diag(DD\trans)$ \COMMENT{Pre-cache values}
    \item $Z \leftarrow D, \quad \Omega \leftarrow \zero, \quad \rho \leftarrow 1$ \COMMENT{Initialize parameters}
    \FOR{$t = 0, 1, 2, \cdots$}
    \item $C \leftarrow \text{ridge}(D, \rho, DX + \rho(Z - \Omega), (1 + \rho^{-1}\Delta)^{-1})$   \COMMENT{Encode data}
    \item $Z \leftarrow g(C + \Omega)$                                  \COMMENT{Regularize solutions}
    \item $\Omega \leftarrow \Omega + C - Z$                            \COMMENT{Update residual}
    \item Optional: update $\rho$
    \ENDFOR
    \item $S_i \leftarrow \text{reshape}(C_{\cdot i})$ \COMMENT{Reshape each encoding}
    \item $F \leftarrow \text{reshape}(D)$ \COMMENT{columnize the dictionary}
    \item $F_i \leftarrow F,\quad E_i \leftarrow \zero, \quad \rho \leftarrow 1$ \COMMENT{Initialize parameters}
    \item $SX_i \leftarrow (S_i)\trans X_{\cdot i}, \quad \sigma_i \leftarrow \diag(S_i S_i\trans)$ \COMMENT{Pre-cache values}
    \FOR{$t = 0, 1, 2, \cdots$} 
    \item $F_i \leftarrow \text{ridge}(S_i, \rho, SX_i + \rho(F - E_i), (1 + \rho^{-1}\sigma_i)^{-1})$ \COMMENT{Optimize each point's codebook (parallel)}
    \item $F \leftarrow \text{proj}(\sum_i F_i + E_i)$ \COMMENT{Project combined codebooks onto feasible set}
    \item $E_i \leftarrow E_i + F_i - F$ \COMMENT{Update residual}
    \item Optional: update $\rho$
    \ENDFOR
    \item $D \leftarrow \text{reshape}(F)$
\ENDFOR
\RETURN $\text{reshape}([I, \quad jI] D_{\cdot dm})$
\end{algorithmic}
\end{algorithm}

\end{document}
